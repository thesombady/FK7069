\documentclass{article}
\usepackage{a4wide}
\usepackage[utf8]{inputenc}
\usepackage{amsmath}
\usepackage{mathtools}
\usepackage{amssymb}
\usepackage[english]{babel}
\usepackage{mdframed}
\usepackage{systeme,}
\usepackage{lipsum}
\usepackage{relsize}
\usepackage{caption}
\usepackage{tikz}
\usepackage{tikz-3dplot}
\usetikzlibrary{shapes.geometric}
\usepackage{pgfplots}
\pgfplotsset{compat=newest}%1.7}
\usepackage{harpoon}%
\usepackage{graphicx}
\usepackage{wrapfig}
\usepackage{subcaption}
\usepackage{authblk}
\usepackage{float}
\usepackage{listings}
\usepackage{xcolor}
\usepackage{chngcntr}
\usepackage{comment}
\usepackage{commath}
\usepackage{hyperref}%Might remove, adds link to each reference
\usepackage{url}
\usepackage{calligra}

% Commands

\newcommand{\w}{\omega}
\newcommand{\curl}[1]{\mathbf{\nabla}\times \mathbf{#1}}
\newcommand{\grad}{\mathbf{\nabla}}
\newcommand{\dive}[1]{\mathbf{\nabla}\cdot \mathbf{#1}}
%\newcommand{\crr}{\mathfrak{r}}
\newcommand{\res}[2]{\text{Res}(#1,#2)}
\newcommand{\laplace}{\nabla^2}
\newcommand{\trace}{\text{Tr}}

% Special character commands
\DeclareMathAlphabet{\mathcalligra}{T1}{calligra}{m}{n}
\DeclareFontShape{T1}{calligra}{m}{n}{<->s*[2.2]callig15}{}
\newcommand{\crr}{\mathcalligra{r}\,}
\newcommand{\boldscriptr}{\pmb{\mathcalligra{r}}\,}

\title{Handin 5}
\author{Author : Andreas Evensen}
\date{Date: \today}
\definecolor{codegreen}{rgb}{0,0.6,0}
\definecolor{codegray}{rgb}{0.5,0.5,0.5}
\definecolor{codepurple}{rgb}{0.58,0,0.82}
\definecolor{backcolour}{rgb}{0.95,0.95,0.92}

\lstdefinestyle{mystyle}{
    backgroundcolor=\color{backcolour},   
    commentstyle=\color{codegreen},
    keywordstyle=\color{magenta},
    numberstyle=\tiny\color{codegray},
    stringstyle=\color{codepurple},
    basicstyle=\ttfamily\footnotesize,
    breakatwhitespace=false,         
    breaklines=true,                 
    captionpos=b,                    
    keepspaces=true,                 
    numbers=left,                    
    numbersep=5pt,                  
    showspaces=false,                
    showstringspaces=false,
    showtabs=false,                  
    tabsize=2
}

\lstset{style=mystyle}

\begin{document}

\maketitle

\section*{Key takeaways from Video 'Low Reynolds number'.}
While decreasing the Reynolds number, $Re$, the motion of the flow becomes less dependent on the inertia from the fluid.
Therefore, the indication of $Re$ eliminates the inertial term compared to the viscous term. When the Reynolds number is small, the fluid can withstand both tensile and tangential stresses.

\vspace{0.7cm}\noindent
In doing lubrication, we limit the contact area between two surfaces by a medium, such medium could be air, or a highly viscous fluid of which limits the contact area;
there exist a flow between the two surfaces that allows 'slipping' and thus friction is reduced. This is valid for small Reynolds number, and this is called narrow channel. 
This also implies laminar flow of the fluid between the two surfaces; smooth paths of the fluid particles in the flow.
The Navier-Stokes equation is non-linear partial-differential equation, coupled with the continuity equation,
\begin{align}
    \partial_t\mathbf{v} + \left(\mathbf{v}\cdot\vec{\nabla}\right)\mathbf{v} &= - \frac{\vec{\nabla}p }{\varrho} - \vec{\nabla}\psi.\label{eq: Navier-Stokes}
\end{align}In the cases of low Reynolds number, this equation simplifies to the Stokes equation,
\begin{align}
    \partial_t\mathbf{v} &= \nu\laplace\mathbf{v} - \vec{\nabla}p.\label{eq: Stokes}
\end{align}
In the case of low Reynolds number, and if within the flow the exists a non-rigid body, the Stokes equation becomes reversible, i.e. no direction of time.
This implies that the direction of the flow is independent of the time, i.e. that the flow is time-independent.
This however, can only occur when the action that causes the flow is reversible, such as the case when a fluid is driven by two plates moving in the opposite direction. 
In the case of a rigid-body of which is flexible, the flow is not reversible, and the direction of the flow is dependent on the time, however the flow is still laminar.

\vspace{0.7cm}\noindent
\textbf{Sedimentation:} When a body is dropped into a fluid of low Reynolds number, the body will fall with a constant velocity, $v$, called the terminal velocity.
The position of at which the body is dropped plays into a role of the terminal velocity, as the body will fall faster if it is falling far away from the boundaries of the fluid.
This is due to flow being zero at the boundaries. Hence, if a layer of dust is placed at the top of a low Reynolds number fluid, the dust-particles with travel down the fluid and create a sediment in the bottom of the fluid.
Moreover, the dust layer will together travel with a constant velocity, $v_{\text{sed}}$, but each individual particle will travel with a velocity such that the layer has a constant velocity.
The velocity of the sediment is proportionally to the number of particles traveling, thus if there are a greater number of particles, the sediment will travel faster.

\vspace{0.7cm}\noindent
\textbf{Hele-Shaw:} Hele-Shaw apparatus is a narrow channel of which is filled with a low Reynolds number fluid, as the fluid flows through the channel. The apparatus allows for the study of the streamlines in the fluid.



\end{document}
 
